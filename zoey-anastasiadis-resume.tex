%-----------------------------------------------------------------------------------------------------------------------------------------------%
%	The MIT License (MIT)
%
%	Copyright (c) 2021 Jitin Nair
%
%	Permission is hereby granted, free of charge, to any person obtaining a copy
%	of this software and associated documentation files (the "Software"), to deal
%	in the Software without restriction, including without limitation the rights
%	to use, copy, modify, merge, publish, distribute, sublicense, and/or sell
%	copies of the Software, and to permit persons to whom the Software is
%	furnished to do so, subject to the following conditions:
%	
%	THE SOFTWARE IS PROVIDED "AS IS", WITHOUT WARRANTY OF ANY KIND, EXPRESS OR
%	IMPLIED, INCLUDING BUT NOT LIMITED TO THE WARRANTIES OF MERCHANTABILITY,
%	FITNESS FOR A PARTICULAR PURPOSE AND NONINFRINGEMENT. IN NO EVENT SHALL THE
%	AUTHORS OR COPYRIGHT HOLDERS BE LIABLE FOR ANY CLAIM, DAMAGES OR OTHER
%	LIABILITY, WHETHER IN AN ACTION OF CONTRACT, TORT OR OTHERWISE, ARISING FROM,
%	OUT OF OR IN CONNECTION WITH THE SOFTWARE OR THE USE OR OTHER DEALINGS IN
%	THE SOFTWARE.
%	
%
%-----------------------------------------------------------------------------------------------------------------------------------------------%

%----------------------------------------------------------------------------------------
%	DOCUMENT DEFINITION
%----------------------------------------------------------------------------------------

% article class because we want to fully customize the page and not use a cv template
\documentclass[a4paper,10pt]{article}

%----------------------------------------------------------------------------------------
%	FONT
%----------------------------------------------------------------------------------------

% % fontspec allows you to use TTF/OTF fonts directly
% \usepackage{fontspec}
% \defaultfontfeatures{Ligatures=TeX}

% % modified for ShareLaTeX use
% \setmainfont[
% SmallCapsFont = Fontin-SmallCaps.otf,
% BoldFont = Fontin-Bold.otf,
% ItalicFont = Fontin-Italic.otf
% ]
% {Fontin.otf}

%----------------------------------------------------------------------------------------
%	PACKAGES
%----------------------------------------------------------------------------------------
\usepackage{url}
\usepackage{parskip} 	

%other packages for formatting
\RequirePackage{color}
\RequirePackage{graphicx}
%\RequirePackage{tikz}
\usepackage[usenames,dvipsnames]{xcolor}
\usepackage[scale=0.9]{geometry}

%tabularx environment
\usepackage{tabularx}

%for lists within experience section
\usepackage{enumitem}

% centered version of 'X' col. type
\newcolumntype{C}{>{\centering\arraybackslash}X} 

%to prevent spillover of tabular into next pages
\usepackage{supertabular}
\usepackage{tabularx}
\newlength{\fullcollw}
\setlength{\fullcollw}{0.47\textwidth}

%custom \section
\usepackage{titlesec}				
\usepackage{multicol}
\usepackage{multirow}

%CV Sections inspired by: 
%http://stefano.italians.nl/archives/26
\titleformat{\section}{\Large\scshape\raggedright}{}{0em}{}[\titlerule]
\titlespacing{\section}{0pt}{6pt}{8pt}

%for publications
%\usepackage[style=authoryear,sorting=ynt, maxbibnames=2]{biblatex}

%Setup hyperref package, and colours for links
\usepackage[unicode, draft=false]{hyperref}
\definecolor{linkcolour}{rgb}{0,0.2,0.6}
\hypersetup{colorlinks,breaklinks,urlcolor=linkcolour,linkcolor=linkcolour}
%\addbibresource{citations.bib}
%\setlength\bibitemsep{1em}

%for social icons
\usepackage{fontawesome5}

%debug page outer frames
%\usepackage{showframe}

% tag (framed word or phrase)
\RequirePackage{tikz}
\colorlet{body}{black!80!white}
\newcommand{\tag}[1]{%
  \tikz[baseline]\node[anchor=base,draw=body!30,rounded corners,inner xsep=1ex,inner ysep =0.75ex,text height=1.5ex,text depth=.25ex]{#1};
}

%----------------------------------------------------------------------------------------
%	BEGIN DOCUMENT
%----------------------------------------------------------------------------------------
\begin{document}

% non-numbered pages
\pagestyle{empty} 

%----------------------------------------------------------------------------------------
%	TITLE
%----------------------------------------------------------------------------------------

\begin{tabularx}{\linewidth}{@{} C @{}}
\Huge{Zoey Anastasiadis} \\[7.0pt]
\href{mailto:zcanast@gmail.com}{\raisebox{-0.05\height}\faEnvelope \ zcanast@gmail.com} \ $|$ \ 
\href{tel:+4432559921}{\raisebox{-0.05\height}\faMobile \ 443.255.9921} \ $|$ \ 
\href{https://github.com/zcanast}{\raisebox{-0.05\height}\faGithub\ zcanast} \ $|$ \ 
\href{https://linkedin.com/in/zoey-anastasiadis}{\raisebox{-0.05\height}\faLinkedin\ zoey-anastasiadis} \ $|$ \ 
\href{https://zcanast.github.io/portfolio}{\raisebox{-0.05\height}\faGlobe \ zcanast.github.io/portfolio} \\
\end{tabularx}

%----------------------------------------------------------------------------------------
% EXPERIENCE SECTIONS
%----------------------------------------------------------------------------------------

%Experience
\section{Work Experience}

\begin{tabularx}{\linewidth}{ @{}l r@{} }
\textbf{Stanley Leadership Program - Information Technology} &  \hfill July 2023 - present \\ 
Stanley Black \& Decker | Remote \\[3.75pt]
\multicolumn{2}{@{}X@{}}{
\begin{minipage}[t]{\linewidth}
    \begin{itemize}[nosep, leftmargin=1em, itemsep=3pt]
        \item[--] 2 year long rotational leadership program with 4 different 6 month long IT rotations\\
        %\item[--] Committees: Engagement (Happy Hour and Workplace Challenges) (aims to connect global SLP’s and alumni through a diverse range of experiences and networking opportunities in and out of the workplace) and Sustainable Development Committee (Foster the Integration Of The Sustainable Development Goals into the SLP Community)\\
        %\item[--] Annual fall and spring trainings: End User Obsession & Innovation (won team project), Operational Excellence\\
        %\item[--] Lean project: As an SLP you will identify and complete a project that improves a process in your workplace. This program requirement allows you to demonstrate your project management and presentation skills along with your lean concept knowledge. Following the implementation of your project, you will report out to the SLP Community & our internal Lean Council experts.\\
        %\item[--] quick learner to switch from rotations and different managers and projects etc.

        %\item[--] Partner \& Value Management -- Portfolio Management \& Process -- IT Process

        % \begin{tabularx}{\linewidth}{ @{}l r@{} }
        %     \textbf{4th Rotation: } &  \hfill January 2025 - present \\ 
        %     % Remote \\[3.75pt]
        %     \multicolumn{2}{@{}X@{}}{
        %     \begin{minipage}[t]{\linewidth}
        %         \begin{itemize}[nosep, leftmargin=1em, itemsep=3pt]
        %            \item[--] Skills used: 
        %         \end{itemize}
        %         \end{minipage}
        %     }\
        %     \end{tabularx}
            
         \begin{tabularx}{\linewidth}{ @{}l r@{} }
             \textbf{3rd Rotation: Data Architect, Enterprise Data - Platform Engineering} &  \hfill July 2024 - present \\ 
             % Remote \\[3.75pt]
             \multicolumn{2}{@{}X@{}}{
             \begin{minipage}[t]{\linewidth}
                 \begin{itemize}[nosep, leftmargin=1em, itemsep=3pt]
                    \item[--] Skills used: Jira, Snowflake, SQL, Python, dbt Cloud, Power BI, Power Query, AWS (WorkSpaces, Lamba, CloudWatch, Simple Notification Service, CloudShell)
                    \item[--] Calculated 8 KPIs through Snowflake SQL scripts and Python scripts calling the dbt Cloud API
                    \item[--] Created a Power BI dashboard of the KPIs for leadership to visualize usage trends
                    \item[--] Compiled a document with the process for engaging support (opening a ticket) for each of our vendors
                    \item[--] Evaluated AWS WorkSpace with Lamda (for triggers) to find WorkSpaces not being used for more than 90 days, saving \$15000 per year in deleted workspaces
                 \end{itemize}
                 \end{minipage}
             }\
             \end{tabularx}

 \begin{tabularx}{\linewidth}{ @{}l r@{} }
            \multicolumn{2}{@{}X@{}}{
            \begin{minipage}[t]{\linewidth}
                \end{minipage}
            }\
            \end{tabularx}
            
        \begin{tabularx}{\linewidth}{ @{}l r@{} }
            \textbf{2nd Rotation: Architecture Process Engineer, Enterprise Architecture} &  \hfill January 2024 - July 2024 \\ 
            % Remote \\[3.75pt]
            \multicolumn{2}{@{}X@{}}{
            \begin{minipage}[t]{\linewidth}
                \begin{itemize}[nosep, leftmargin=1em, itemsep=3pt]
                   \item[--] Skills used: Ardoq, Azure AI, Power BI, Python (pandas, openpyxl, openai, box, langchain, json), Excel (pivot tables, graphs, complex nested formulas (IF, OR, AND, COUNTIF, XLOOKUP))
                   \item[--] Documented the current state architecture for SBD capability areas using our Enterprise Architecture Tool, Ardoq
                   \item[--] Led a campaign to establish data quality metrics by analyzing the quality levels of existing application and application instance data hosted in Ardoq and then creating data quality scores and a presentation plan for these scores
                   \item[--] Collaborated with IT stakeholders on a data certification initiative in which we created a simple toolset and process for capability teams to regularly update and certify their data - created surveys and broadcasts in Ardoq, led training demos, and drove the process mapping
                   \item[--] Developed a Python script which reads 2500 application names from an Excel file and uses Open AI to generate a list of corresponding app descriptions, vendor names, and vendor URLs, bringing the completion score from 30\% to 63\%
                   \item[--] Implemented a generative AI automation solution using Python to seamlessly retrieve 40,000 product manual PDFs from Box, extract metadata using Open AI's GPT, integrate it into an Excel template, and upload both the manuals and Excel file with metadata to Bynder, resulting in 650 productivity hours saved
                   \item[--] Participated in an initiative to disseminate communications to the IT organization on generative AI prompt engineering
                   \item[--] Designed reporting dashboards about survey progress using Excel and Power BI
                   \item[--] Improved leadership and presentation skills through status meetings with project stakeholders
                \end{itemize}
                \end{minipage}
            }\
            \end{tabularx}

 \begin{tabularx}{\linewidth}{ @{}l r@{} }
            \multicolumn{2}{@{}X@{}}{
            \begin{minipage}[t]{\linewidth}
                \end{minipage}
            }\
            \end{tabularx}

        \begin{tabularx}{\linewidth}{ @{}l r@{} }
            \textbf{1st Rotation: Process Analyst, IT Process} &  \hfill July 2023 - January 2024 \\ 
            % Remote \\[3.75pt]
            \multicolumn{2}{@{}X@{}}{
            \begin{minipage}[t]{\linewidth}
                \begin{itemize}[nosep, leftmargin=1em, itemsep=3pt]
                   %\item[--] Using Excel and Power BI to create statistics and visualizations to analyze what could be done better in projects. What are the trouble areas and stress points? What are teams struggling with? I am working to find out what the data is telling us (analysis). Then I will create a repeatable process (monthly, quarterly, or annually). I will create a list of possible opportunities based on this data to present to change champions. I will create a feedback form or comment box for teams to leave feedback about where the issues are and the process I am creating will be how to review this feedback and to create a list to go after in the next year/quarter/etc. 
                    %\item[--] Performed data analysis of existing lessons learned data using to identify trends and opportunities for improvement; performed a read out of process opportunities based on lessons learned data; went through several iterations to determine which results were the most valuable and how to present them and to ensure the most effective and time efficient tools were used (and would be used in the go-forward process)
                    %\item[--] proposed go-forward process for lessons learned data; created a LEAN repeatable process to perform analysis of lessons learned data for regular reports biannually
                    %\item[--] Used data to initiate data driven conversations; presentations on lessons learned findings and opportunities to change champions, leadership staff meetings, and project management office; used data to highlight areas for improvement; allowed me to communicate with many different groups; expanded upon my abilities to report out, write a powerpoint presentation for leadership and for people with no prior knowledge of the project and data, learned how to take constructive criticism and work it into future iternations of a project and presentation; how to make data understandable and answer the questions "what does the data tell us" and "what is the 'so what' factor of the data"
                    %\item[--] Project managers submit lessons learned data as they close out projects. The team has been collecting this data since June 2021 and had never done a cohesive analysis of it. I used Excel as my main tool to find trends in the data, with help from Miro and Azure AI. We found that planning and vendors were the two areas with the biggest room for improvement. The biggest struggle with this project was learning how to analyze text data; I used Azure AI to parse the data in a time efficient manner along with complex, nested Excel formulas and Excel macros/VBA. After completing the analysis, I presented my findings to multiple groups of leaders & stakeholders. \\
                    %\item[--] During the process of analyzing the lessons learned data, I explored different tools and methods of analysis to propose a go-forward process for repeating the analysis on a regular basis. I went through several iterations to determine which results were the most valuable and how to achieve them in an effective and time efficient manner. I was able to create a master Excel file which is connected to the SharePoint list where the lessons learned data lives. The Excel sheet updates as the SharePoint list is updated. I proposed a biannual analysis of this data, and the Excel sheet will simply need to be refreshed with the new data and the existing analyses will update.
                    %\item[--] Read out the results and opportunities found from lessons learned analysis to the leaders and stakeholders; I used data to highlight areas for improvement. This allowed me to communicate with many different groups of people and taught me how to write/create a PowerPoint presentation for leadership and for people with no prior knowledge of the subject/data. This pushed me to think about how I can make data understandable and answer the questions "what does the data tell us?" and "what is the 'so what' factor of the data?".
                    \item[--] Skills used: Power BI, PowerPoint, Miro, Azure AI, Power Automate, SharePoint (lists), Excel (pivot tables, graphs (Paretos, Gantt charts), complex nested formulas (IF, OR, AND, SEARCH, FIND, ISNUMBER, COUNTIF, XLOOKUP), Power Query, macros/VBA)
                    \item[--] Conducted comprehensive data analysis of existing lessons learned data using Excel, Power BI, Azure AI, and Miro to identify trends and opportunities for process and project improvements
                    \item[--] Proposed a LEAN repeatable process for regular analysis of lessons learned data, resulting in biannual reports that provided valuable insights for optimizing processes and projects
                    \item[--] Determined the biggest project change reasons (PCRs) using Excel analysis of PPM data, enabling proactive measures to mitigate risks and enhance project outcomes
                    \item[--] Performed data analysis tasks as required, employing various tools and techniques to extract meaningful insights and support decision-making processes
                    \item[--] Utilized data-driven presentations to initiate and facilitate discussions on findings, opportunities, and actionable recommendations, fostering a culture of evidence-based decision-making.
                    %Used data to initiate data driven conversations through presentations on findings and opportunities
                \end{itemize}
                \end{minipage}
            }\
            \end{tabularx}

    \end{itemize}
    \end{minipage}
}\
\end{tabularx}
    
\begin{tabularx}{\linewidth}{ @{}l r@{} }
\textbf{iSchool Undergraduate Assistant for Research Programs} &  \hfill January 2022 - May 2023 \\ 
University of Maryland, College of Information Studies | College Park, MD \\[3.75pt]
\multicolumn{2}{@{}X@{}}{
\begin{minipage}[t]{\linewidth}
    \begin{itemize}[nosep,after=\strut, leftmargin=1em, itemsep=3pt]
        \item[--] Facilitated research-related communications for the Associate Dean of Research through various channels
        \item[--] Gathered, analyzed, and interpreted research-related data, and prepared Excel reports for effective decision-making
        \item[--] Updated websites using WordPress and maintained high-quality online resources through regular quality assurance checks
        \item[--] Executed research-related events and projects and configured video conferencing for meetings and presentations
    \end{itemize}
    \end{minipage}
}\
\end{tabularx}


    
\begin{tabularx}{\linewidth}{ @{}l r@{} }
\textbf{Technology Intern} &  \hfill June 2022 - August 2022 \\
Stanley Black \& Decker | Remote \\[3.75pt]
\multicolumn{2}{@{}X@{}}{
\begin{minipage}[t]{\linewidth}
    \begin{itemize}[nosep,after=\strut, leftmargin=1em, itemsep=3pt]
        \item[--] Coordinated and managed 9 training sessions for 450+ IT employees, utilizing Excel to monitor enrollment and attendance
        \item[--] Produced engaging explainer videos using Powtoon for each IT role to facilitate employee onboarding and training
        \item[--] Designed Mentimeter questionnaires for training sessions and analyzed live polling results to enhance training effectiveness
        \item[--] Collected, interpreted, and presented data from questionnaires in a visually compelling format to create reports
        \item[--] Maintained SharePoint site to disseminate communications throughout the organization
    \end{itemize}
    \end{minipage}
}\
\end{tabularx}

%----------------------------------------------------------------------------------------
%	EDUCATION
%----------------------------------------------------------------------------------------
\section{Education}
\begin{tabularx}{\linewidth}{@{}l X@{}}	

\textbf{University of Maryland} | College Park, MD &  \hfill \normalsize May 2023 \\

Bachelor of Science in Information Science & \hfill \\ %GPA: 3.62 \\ 

\end{tabularx}

%----------------------------------------------------------------------------------------
%	CERTIFICATIONS
%----------------------------------------------------------------------------------------
% \section{Certifications}
% \begin{tabularx}{\linewidth}{@{}l X@{}}	

% \textbf{Career Essentials in Data Analysis} | Microsoft and LinkedIn &  \hfill \normalsize July 2023 \\

% \end{tabularx}

%----------------------------------------------------------------------------------------
%	SKILLS
%----------------------------------------------------------------------------------------
\section{Skills}
\begin{tabularx}{\linewidth}{@{}l X@{}}

%\textbf{Languages} & \normalsize{Python, Java, SQL, R, HTML/CSS, JavaScript, MATLAB}\\
%\textbf{Technologies}  &  \normalsize{Excel, GitHub, Microsoft Office, Google Suite, WordPress, Tableau, Power BI, Gephi, Adobe Premiere Pro, Adobe Photoshop, Figma, Canva}\\ 
%\textbf{Environments}  &  \normalsize{Visual Studio Code, Eclipse, MySQL, RStudio, Jupyter Notebooks}\\

\textbf{Languages} & \normalsize
\tag{Python}
\tag{Java}
\tag{R} 
\tag{SQL}
\tag{HTML/CSS}
\tag{JavaScript}
\tag{Visual Basic}
\tag{MATLAB}
\\

\textbf{Technologies}  &  \normalsize{
\tag{Excel}
\tag{GitHub}
\tag{Microsoft Office}
\tag{Google Suite}
\tag{WordPress}
\tag{Tableau}
\tag{Power BI}
\tag{SharePoint}
\tag{Power Automate}
\tag{Adobe Premiere Pro}
\tag{Adobe Photoshop}
\tag{Canva}
\tag{Miro}
\tag{Open AI}
\tag{Figma}
\tag{Gephi}
\tag{Powtoon}
}\\

\textbf{Environments}  &  \normalsize{
\tag{Visual Studio Code}
\tag{Eclipse}
\tag{MySQL}
\tag{RStudio}
\tag{Jupyter Notebooks}
}\\

\end{tabularx}

\vfill
\center{\footnotesize Last updated: \today}

\end{document}
