%-----------------------------------------------------------------------------------------------------------------------------------------------%
%	The MIT License (MIT)
%-----------------------------------------------------------------------------------------------------------------------------------------------%

%----------------------------------------------------------------------------------------
%	DOCUMENT DEFINITION
%----------------------------------------------------------------------------------------

\documentclass[a4paper,10pt]{article}

%----------------------------------------------------------------------------------------
%	PACKAGES
%----------------------------------------------------------------------------------------

\usepackage{url}
\usepackage{parskip} 	
\usepackage{color}
\usepackage{graphicx}
\usepackage[scale=0.9]{geometry}
\usepackage{tabularx}
\usepackage{enumitem}
\usepackage{supertabular}
\usepackage{titlesec}				
\usepackage{multicol}
\usepackage{multirow}
\usepackage{hyperref}
\usepackage{fontawesome5}
\usepackage{tikz}
\usepackage{xcolor}
\usepackage{mathptmx} % Use Times-like font

% Setup hyperref package, and colours for links
\definecolor{linkcolour}{rgb}{0,0.2,0.6}
\hypersetup{
    colorlinks=true,
    breaklinks=true,
    urlcolor=linkcolour,
    linkcolor=linkcolour,
    pdftitle={Zoey Anastasiadis Resume},
    pdfauthor={Zoey Anastasiadis},
    pdfsubject={Resume}
}

% Custom \section
\titleformat{\section}{\Large\scshape\raggedright\color{RoyalBlue}}{}{0em}{}[\titlerule]
\titlespacing{\section}{0pt}{10pt}{12pt} % Increased spacing before and after sections

% Define centered column type for tabularx
\newcolumntype{C}{>{\centering\arraybackslash}X}

% Tag (framed word or phrase)
\colorlet{body}{black!80!white}
\newcommand{\tag}[1]{%
  \tikz[baseline]\node[anchor=base,draw=body!30,rounded corners,inner xsep=1ex,inner ysep =0.75ex,text height=1.5ex,text depth=.25ex]{#1};
}

%----------------------------------------------------------------------------------------
%	BEGIN DOCUMENT
%----------------------------------------------------------------------------------------
\begin{document}

% Non-numbered pages
\pagestyle{empty} 

%----------------------------------------------------------------------------------------
%	TITLE
%----------------------------------------------------------------------------------------

\begin{tabularx}{\linewidth}{@{} C @{}}
\Huge{Zoey Anastasiadis} \\[7.0pt]
\href{mailto:zcanast@gmail.com}{\raisebox{-0.05\height}\faEnvelope \ zcanast@gmail.com} \ $|$ \ 
\href{tel:+4432559921}{\raisebox{-0.05\height}\faMobile \ 443.255.9921} \ $|$ \ 
\href{https://github.com/zcanast}{\raisebox{-0.05\height}\faGithub\ zcanast} \ $|$ \ 
\href{https://linkedin.com/in/zoey-anastasiadis}{\raisebox{-0.05\height}\faLinkedin\ zoey-anastasiadis} \ $|$ \ 
\href{https://zcanast.github.io/portfolio}{\raisebox{-0.05\height}\faGlobe \ zcanast.github.io/portfolio} \\
\end{tabularx}

%----------------------------------------------------------------------------------------
% EXPERIENCE SECTIONS
%----------------------------------------------------------------------------------------

% Experience
\section{Work Experience}

\begin{tabularx}{\linewidth}{ @{}l r@{} }
\textbf{Stanley Leadership Program - Information Technology} &  \hfill July 2023 - Present \\ 
Stanley Black \& Decker | Remote \\[3.75pt]
\multicolumn{2}{@{}X@{}}{
\begin{minipage}[t]{\linewidth}
    \begin{itemize}[nosep, leftmargin=1em, itemsep=1pt, label=\textendash]
        \item 2-year long rotational leadership program with 4 different 6-month long IT rotations
        \item Co-lead of Sustainable Development Committee, member of Engagement Committee
    \end{itemize}
    \end{minipage}
}\
\end{tabularx}

% Indented Rotations with reduced extra space
\begin{tabularx}{\linewidth}{ @{}l r@{} }
\hspace{1em}\textbf{4th Rotation: GenAI App Architecture} &  \hfill January 2025 - Present \\  
\multicolumn{2}{@{}X@{}}{
\begin{minipage}[t]{\linewidth}
    \begin{itemize}[nosep, leftmargin=2em, itemsep=3pt, label=\textendash]
        \item Skills used: GenAI, prompt engineering, app architecture, SQL, Excel (pivot tables), Snowflake, YAML
        \item Contribute to the optimization of Caspian Data Architecture to maximize business value, ensuring all Data, Artificial Intelligence (AI), and Application (App) assets are well-documented and discoverable, with 100\% of Semantic Models certified and documented in Ataccama
        \item Ensure adoption of Data, AI, and Robotic Process Automation/Business Process Management (RPA/BPM) for delivered assets, and deliver at least 5 Generative AI (Gen AI) applications, including Zenlytic and ChatSBD+, with a focus on developing an internal pro-code agentic system framework
        \item Drive the adoption of AI-powered Software Development Life Cycle (SDLC) for delivery, launch an agentic framework and front-end user platform, enable Zenlytic adoption for 80\% of certified Business Ready Zone assets, and establish tools for citizen AI development
        \item Create content to drive AI literacy and responsible AI education, and set up multiple channels for AI upskilling, including hosting monthly lunch and learn sessions with an average attendance of 50+ participants
    \end{itemize}
    \end{minipage}
}\
\end{tabularx}

\vspace{0.5em} % Reduced space between experiences

\begin{tabularx}{\linewidth}{ @{}l r@{} }
\hspace{1em}\textbf{3rd Rotation: Data Architect, Enterprise Data - Platform Engineering} &  \hfill July 2024 - January 2025 \\ 
\multicolumn{2}{@{}X@{}}{
\begin{minipage}[t]{\linewidth}
    \begin{itemize}[nosep, leftmargin=2em, itemsep=3pt, label=\textendash]
        \item Skills used: Jira, Snowflake, SQL, Python, dbt Cloud, Power BI, Power Query, Power Automate, AWS (WorkSpaces, Lambda, CloudWatch, Simple Notification Service, CloudShell)
        \item Calculated 8 KPIs through Snowflake SQL scripts and Python scripts calling the dbt Cloud API
        \item Created a Power BI dashboard of the KPIs for leadership to visualize usage trends, saving ~40 hours per month of data collection
        \item Compiled a document with the process for engaging support (opening a ticket) for each of our vendors
        \item Evaluated AWS WorkSpaces through a Lambda function to find and delete WorkSpaces not being used for more than 90 days, saving \$20,000 per year
        \item Used Lambda function in AWS to create a function to automate the key-pair rotation process for Snowflake
    \end{itemize}
    \end{minipage}
}\
\end{tabularx}

\vspace{0.5em} % Reduced space between experiences

\begin{tabularx}{\linewidth}{ @{}l r@{} }
\hspace{1em}\textbf{2nd Rotation: Architecture Process Engineer, Enterprise Architecture} &  \hfill January 2024 - July 2024 \\ 
\multicolumn{2}{@{}X@{}}{
\begin{minipage}[t]{\linewidth}
    \begin{itemize}[nosep, leftmargin=2em, itemsep=3pt, label=\textendash]
        \item Skills used: Ardoq, Azure AI, Power BI, Python (pandas, openpyxl, openai, box, langchain, json), Excel (pivot tables, graphs, complex nested formulas (IF, OR, AND, COUNTIF, XLOOKUP))
        \item Document the current state architecture for SBD capability areas using our Enterprise Architecture Tool, Ardoq
        \item Lead a campaign to establish data quality metrics by analyzing the quality levels of existing application and application instance data hosted in Ardoq and then creating data quality scores and a presentation plan for these scores
        \item Collaborate with IT stakeholders on a data certification initiative in which we create a simple toolset and process for capability teams to regularly update and certify their data - create surveys and broadcasts in Ardoq, lead training demos, and drive the process mapping
        \item Develop a Python script which reads 2500 application names from an Excel file and uses Open AI to generate a list of corresponding app descriptions, vendor names, and vendor URLs, bringing the completion score from 30\% to 63\%
        \item Implement a generative AI automation solution using Python to seamlessly retrieve 40,000 product manual PDFs from Box, extract metadata using Open AI's GPT, integrate it into an Excel template, and upload both the manuals and Excel file with metadata to Bynder, resulting in 650 productivity hours saved
        \item Participate in an initiative to disseminate communications to the IT organization on generative AI prompt engineering
        \item Design reporting dashboards about survey progress using Excel and Power BI
        \item Improve leadership and presentation skills through status meetings with project stakeholders
    \end{itemize}
    \end{minipage}
}\
\end{tabularx}

\vspace{0.5em} % Reduced space between experiences

\begin{tabularx}{\linewidth}{ @{}l r@{} }
\hspace{1em}\textbf{1st Rotation: Process Analyst, IT Process} &  \hfill July 2023 - January 2024 \\ 
\multicolumn{2}{@{}X@{}}{
\begin{minipage}[t]{\linewidth}
    \begin{itemize}[nosep, leftmargin=2em, itemsep=3pt, label=\textendash]
        \item Skills used: Power BI, PowerPoint, Miro, Azure AI, Power Automate, SharePoint (lists), Excel (pivot tables, graphs (Paretos, Gantt charts), complex nested formulas (IF, OR, AND, SEARCH, FIND, ISNUMBER, COUNTIF, XLOOKUP), Power Query, macros/VBA)
        \item Conduct comprehensive data analysis of existing lessons learned data using Excel, Power BI, Azure AI, and Miro to identify trends and opportunities for process and project improvements
        \item Propose a LEAN repeatable process for regular analysis of lessons learned data, resulting in biannual reports that provide valuable insights for optimizing processes and projects
        \item Determine the biggest project change reasons (PCRs) using Excel analysis of PPM data, enabling proactive measures to mitigate risks and enhance project outcomes
        \item Perform data analysis tasks as required, employing various tools and techniques to extract meaningful insights and support decision-making processes
        \item Utilize data-driven presentations to initiate and facilitate discussions on findings, opportunities, and actionable recommendations, fostering a culture of evidence-based decision-making
    \end{itemize}
    \end{minipage}
}\
\end{tabularx}

%----------------------------------------------------------------------------------------
%	EDUCATION
%----------------------------------------------------------------------------------------
\section{Education}
\begin{tabularx}{\linewidth}{@{}l X@{}}	
\textbf{University of Maryland} | College Park, MD &  \hfill \normalsize May 2023 \\
Bachelor of Science in Information Science & \hfill \\ 
\end{tabularx}

%----------------------------------------------------------------------------------------
%	SKILLS
%----------------------------------------------------------------------------------------
\section{Skills}
\begin{tabularx}{\linewidth}{@{}l X@{}}

\textbf{Languages} & \normalsize
\tag{Python}
\tag{Java}
\tag{R} 
\tag{SQL}
\tag{HTML/CSS}
\tag{JavaScript}
\tag{Visual Basic}
\tag{MATLAB}
\tag{YAML}
\\

\textbf{Technologies}  &  \normalsize{
\tag{Excel}
\tag{AWS}
\tag{GitHub}
\tag{Microsoft Office}
\tag{Google Suite}
\tag{WordPress}
\tag{Tableau}
\tag{Power BI}
\tag{Snowflake}
\tag{Zenlytic}
\tag{Revefi}
\tag{Ataccama}
\tag{dbt Cloud}
\tag{SharePoint}
\tag{Power Automate}
\tag{Power Query}
\tag{Adobe Premiere Pro}
\tag{Adobe Photoshop}
\tag{Canva}
\tag{Jira}
\tag{Miro}
\tag{Open AI}
\tag{Figma}
\tag{Gephi}
\tag{Powtoon}
}\\

\textbf{Environments}  &  \normalsize{
\tag{Visual Studio Code}
\tag{Eclipse}
\tag{MySQL}
\tag{RStudio}
\tag{Jupyter Notebooks}
}\\

\end{tabularx}

\vfill
\center{\footnotesize Last updated: \today}

\end{document}
