%-----------------------------------------------------------------------------------------------------------------------------------------------%
%	The MIT License (MIT)
%
%	Copyright (c) 2021 Jitin Nair
%
%	Permission is hereby granted, free of charge, to any person obtaining a copy
%	of this software and associated documentation files (the "Software"), to deal
%	in the Software without restriction, including without limitation the rights
%	to use, copy, modify, merge, publish, distribute, sublicense, and/or sell
%	copies of the Software, and to permit persons to whom the Software is
%	furnished to do so, subject to the following conditions:
%	
%	THE SOFTWARE IS PROVIDED "AS IS", WITHOUT WARRANTY OF ANY KIND, EXPRESS OR
%	IMPLIED, INCLUDING BUT NOT LIMITED TO THE WARRANTIES OF MERCHANTABILITY,
%	FITNESS FOR A PARTICULAR PURPOSE AND NONINFRINGEMENT. IN NO EVENT SHALL THE
%	AUTHORS OR COPYRIGHT HOLDERS BE LIABLE FOR ANY CLAIM, DAMAGES OR OTHER
%	LIABILITY, WHETHER IN AN ACTION OF CONTRACT, TORT OR OTHERWISE, ARISING FROM,
%	OUT OF OR IN CONNECTION WITH THE SOFTWARE OR THE USE OR OTHER DEALINGS IN
%	THE SOFTWARE.
%	
%
%-----------------------------------------------------------------------------------------------------------------------------------------------%

%----------------------------------------------------------------------------------------
%	DOCUMENT DEFINITION
%----------------------------------------------------------------------------------------

% article class because we want to fully customize the page and not use a cv template
\documentclass[a4paper,10pt]{article}

%----------------------------------------------------------------------------------------
%	FONT
%----------------------------------------------------------------------------------------

% % fontspec allows you to use TTF/OTF fonts directly
% \usepackage{fontspec}
% \defaultfontfeatures{Ligatures=TeX}

% % modified for ShareLaTeX use
% \setmainfont[
% SmallCapsFont = Fontin-SmallCaps.otf,
% BoldFont = Fontin-Bold.otf,
% ItalicFont = Fontin-Italic.otf
% ]
% {Fontin.otf}

%----------------------------------------------------------------------------------------
%	PACKAGES
%----------------------------------------------------------------------------------------
\usepackage{url}
\usepackage{parskip} 	

%other packages for formatting
\RequirePackage{color}
\RequirePackage{graphicx}
%\RequirePackage{tikz}
\usepackage[usenames,dvipsnames]{xcolor}
\usepackage[scale=0.9]{geometry}

%tabularx environment
\usepackage{tabularx}

%for lists within experience section
\usepackage{enumitem}

% centered version of 'X' col. type
\newcolumntype{C}{>{\centering\arraybackslash}X} 

%to prevent spillover of tabular into next pages
\usepackage{supertabular}
\usepackage{tabularx}
\newlength{\fullcollw}
\setlength{\fullcollw}{0.47\textwidth}

%custom \section
\usepackage{titlesec}				
\usepackage{multicol}
\usepackage{multirow}

%CV Sections inspired by: 
%http://stefano.italians.nl/archives/26
\titleformat{\section}{\Large\scshape\raggedright}{}{0em}{}[\titlerule]
\titlespacing{\section}{0pt}{6pt}{8pt}

%for publications
%\usepackage[style=authoryear,sorting=ynt, maxbibnames=2]{biblatex}

%Setup hyperref package, and colours for links
\usepackage[unicode, draft=false]{hyperref}
\definecolor{linkcolour}{rgb}{0,0.2,0.6}
\hypersetup{colorlinks,breaklinks,urlcolor=linkcolour,linkcolor=linkcolour}
%\addbibresource{citations.bib}
%\setlength\bibitemsep{1em}

%for social icons
\usepackage{fontawesome5}

%debug page outer frames
%\usepackage{showframe}

% tag (framed word or phrase)
\RequirePackage{tikz}
\colorlet{body}{black!80!white}
\newcommand{\tag}[1]{%
  \tikz[baseline]\node[anchor=base,draw=body!30,rounded corners,inner xsep=1ex,inner ysep =0.75ex,text height=1.5ex,text depth=.25ex]{#1};
}

%----------------------------------------------------------------------------------------
%	BEGIN DOCUMENT
%----------------------------------------------------------------------------------------
\begin{document}

% non-numbered pages
\pagestyle{empty} 

%----------------------------------------------------------------------------------------
%	TITLE
%----------------------------------------------------------------------------------------

\begin{tabularx}{\linewidth}{@{} C @{}}
\Huge{Zoey Anastasiadis} \\[7.0pt]
\href{mailto:zcanast@gmail.com}{\raisebox{-0.05\height}\faEnvelope \ zcanast@gmail.com} \ $|$ \ 
\href{tel:+4432559921}{\raisebox{-0.05\height}\faMobile \ 443.255.9921} \ $|$ \ 
\href{https://github.com/zcanast}{\raisebox{-0.05\height}\faGithub\ zcanast} \ $|$ \ 
\href{https://linkedin.com/in/zoey-anastasiadis}{\raisebox{-0.05\height}\faLinkedin\ zoey-anastasiadis} \ $|$ \ 
\href{https://zcanast.github.io/portfolio}{\raisebox{-0.05\height}\faGlobe \ zcanast.github.io/portfolio} \\
\end{tabularx}

%----------------------------------------------------------------------------------------
% EXPERIENCE SECTIONS
%----------------------------------------------------------------------------------------

%Experience
\section{Work Experience}

\begin{tabularx}{\linewidth}{ @{}l r@{} }
\textbf{Stanley Leadership Program - Information Technology} &  \hfill July 2023 - present \\ 
Stanley Black \& Decker | New Britain, CT (Remote) \\[3.75pt]
\multicolumn{2}{@{}X@{}}{
\begin{minipage}[t]{\linewidth}
    \begin{itemize}[nosep, leftmargin=1em, itemsep=3pt]
        \item[--] 2 year long rotational leadership program with 4 different 6 month long IT rotations \\
        %\item[--] Partner \& Value Management -- Portfolio Management \& Process -- IT Process

        % \begin{tabularx}{\linewidth}{ @{}l r@{} }
        %     \textbf{4th Rotation: } &  \hfill January 2025 - present \\ 
        %     % Remote \\[3.75pt]
        %     \multicolumn{2}{@{}X@{}}{
        %     \begin{minipage}[t]{\linewidth}
        %         \begin{itemize}[nosep, leftmargin=1em, itemsep=3pt]
        %            \item[--] Skills used: 
        %         \end{itemize}
        %         \end{minipage}
        %     }\
        %     \end{tabularx}
            
        % \begin{tabularx}{\linewidth}{ @{}l r@{} }
        %     \textbf{3rd Rotation: } &  \hfill July 2024 - present \\ 
        %     % Remote \\[3.75pt]
        %     \multicolumn{2}{@{}X@{}}{
        %     \begin{minipage}[t]{\linewidth}
        %         \begin{itemize}[nosep, leftmargin=1em, itemsep=3pt]
        %            \item[--] Skills used: 
        %         \end{itemize}
        %         \end{minipage}
        %     }\
        %     \end{tabularx}

        \begin{tabularx}{\linewidth}{ @{}l r@{} }
            \textbf{2nd Rotation: Architecture Process Engineer, Enterprise Architecture} &  \hfill January 2024 - present \\ 
            % Remote \\[3.75pt]
            \multicolumn{2}{@{}X@{}}{
            \begin{minipage}[t]{\linewidth}
                \begin{itemize}[nosep, leftmargin=1em, itemsep=3pt]
                   \item[--] Skills used: Ardoq, Azure AI, Python (pandas, openpyxl, openai), Excel (pivot tables, graphs, complex nested formulas (IF, OR, AND, COUNTIF, XLOOKUP))
                   \item[--] Gain a broad view of the IT organization and the technology it uses to both support and enable the business
                   %\item[--] Become proficient in Ardoq and Architecture Documentation - document the current state architecture for our capabilities areas by partnering and collaborating with the Enterprise Architects, Capability Owners and Business Owners of each of the applications using our new Enterprise Architecture Tool, Ardoq, to capture application information and architecture diagrams for the existing architectures
                   \item[--] Document the current state architecture for SBD capability areas using our Enterprise Architecture Tool, Ardoq, and by partnering with the Enterprise Architects, Capability Owners and Business Owners of each of the applications
                   %\item[--] Analyzed the quality levels of existing application and application instance data using Ardoq, Excel, and Power BI
                   %\item[--] Created data quality scores for application data and developed a presentation plan for these scores
                   %\item[--] Establish a reusable Metrics platform - develop the Enterprise Architecture Metrics Dashboard by assisting in defining the metrics, helping source the underlying data and develop an on-going maintenance plan
                   \item[--] Led a campaign to establish data quality metrics by analyzing the quality levels of existing application and application instance data hosted in Ardoq and then creating data quality scores and a presentation plan for these scores
                   \item[--] Developed a Python script integrated with Open AI which is used to read an Excel file of application names and generate a list of 2500+ app descriptions, vendor names, and vendor URLs, completion percentage for app descriptions went from 30\% to 63\% after using script
                   %\item[--] Understand Overall Architecture Process and Principles   
                   %\item[--] Establish foundational skills for being a Leader at Stanley Black and Decker
                \end{itemize}
                \end{minipage}
            }\
            \end{tabularx}

        \begin{tabularx}{\linewidth}{ @{}l r@{} }
            \textbf{1st Rotation: Process Analyst, IT Process} &  \hfill July 2023 - January 2024 \\ 
            % Remote \\[3.75pt]
            \multicolumn{2}{@{}X@{}}{
            \begin{minipage}[t]{\linewidth}
                \begin{itemize}[nosep, leftmargin=1em, itemsep=3pt]
                   %\item[--] Using Excel and Power BI to create statistics and visualizations to analyze what could be done better in projects. What are the trouble areas and stress points? What are teams struggling with? I am working to find out what the data is telling us (analysis). Then I will create a repeatable process (monthly, quarterly, or annually). I will create a list of possible opportunities based on this data to present to change champions. I will create a feedback form or comment box for teams to leave feedback about where the issues are and the process I am creating will be how to review this feedback and to create a list to go after in the next year/quarter/etc. 
                    %\item[--] Performed data analysis of existing lessons learned data using to identify trends and opportunities for improvement; performed a read out of process opportunities based on lessons learned data; went through several iterations to determine which results were the most valuable and how to present them and to ensure the most effective and time efficient tools were used (and would be used in the go-forward process)
                    %\item[--] proposed go-forward process for lessons learned data; created a LEAN repeatable process to perform analysis of lessons learned data for regular reports biannually
                    %\item[--] Used data to initiate data driven conversations; presentations on lessons learned findings and opportunities to change champions, leadership staff meetings, and project management office; used data to highlight areas for improvement; allowed me to communicate with many different groups; expanded upon my abilities to report out, write a powerpoint presentation for leadership and for people with no prior knowledge of the project and data, learned how to take constructive criticism and work it into future iternations of a project and presentation; how to make data understandable and answer the questions "what does the data tell us" and "what is the 'so what' factor of the data"
                    %\item[--] Project managers submit lessons learned data as they close out projects. The team has been collecting this data since June 2021 and had never done a cohesive analysis of it. I used Excel as my main tool to find trends in the data, with help from Miro and Azure AI. We found that planning and vendors were the two areas with the biggest room for improvement. The biggest struggle with this project was learning how to analyze text data; I used Azure AI to parse the data in a time efficient manner along with complex, nested Excel formulas and Excel macros/VBA. After completing the analysis, I presented my findings to multiple groups of leaders & stakeholders. \\
                    %\item[--] During the process of analyzing the lessons learned data, I explored different tools and methods of analysis to propose a go-forward process for repeating the analysis on a regular basis. I went through several iterations to determine which results were the most valuable and how to achieve them in an effective and time efficient manner. I was able to create a master Excel file which is connected to the SharePoint list where the lessons learned data lives. The Excel sheet updates as the SharePoint list is updated. I proposed a biannual analysis of this data, and the Excel sheet will simply need to be refreshed with the new data and the existing analyses will update.
                    %\item[--] Read out the results and opportunities found from lessons learned analysis to the leaders and stakeholders; I used data to highlight areas for improvement. This allowed me to communicate with many different groups of people and taught me how to write/create a PowerPoint presentation for leadership and for people with no prior knowledge of the subject/data. This pushed me to think about how I can make data understandable and answer the questions "what does the data tell us?" and "what is the 'so what' factor of the data?".
                    \item[--] Skills used: Power BI, PowerPoint, Miro, Azure AI, Power Automate, SharePoint (lists), Excel (pivot tables, graphs (Paretos, Gantt charts), complex nested formulas (IF, OR, AND, SEARCH, FIND, ISNUMBER, COUNTIF, XLOOKUP), Power Query, macros/VBA)
                    \item[--] Performed data analysis of existing lessons learned data using to identify trends and opportunities for improvement using Excel, Power BI, Miro, Azure AI, and Miro
                    \item[--] Proposed a LEAN repeatable process to perform analysis of lessons learned data for regular reports biannually %Proposed go-forward process for analyzing lessons learned data
                    \item[--] Used data to initiate data driven conversations through presentations on findings and opportunities
                \end{itemize}
                \end{minipage}
            }\
            \end{tabularx}

    \end{itemize}
    \end{minipage}
}\
\end{tabularx}
    
\begin{tabularx}{\linewidth}{ @{}l r@{} }
\textbf{iSchool Undergraduate Assistant for Research Programs} &  \hfill January 2022 - May 2023 \\ 
University of Maryland, College of Information Studies | College Park, MD \\[3.75pt]
\multicolumn{2}{@{}X@{}}{
\begin{minipage}[t]{\linewidth}
    \begin{itemize}[nosep,after=\strut, leftmargin=1em, itemsep=3pt]
        \item[--] Facilitated research-related communications for the Associate Dean of Research through various channels
        \item[--] Gathered, analyzed, and interpreted research-related data, and prepared Excel reports for effective decision-making
        \item[--] Updated websites using WordPress and maintained high-quality online resources through regular quality assurance checks
        \item[--] Executed research-related events and projects and configured video conferencing for meetings and presentations
    \end{itemize}
    \end{minipage}
}\
\end{tabularx}


    
\begin{tabularx}{\linewidth}{ @{}l r@{} }
\textbf{Technology Intern} &  \hfill June 2022 - August 2022 \\
Stanley Black \& Decker | Towson, MD (Remote) \\[3.75pt]
\multicolumn{2}{@{}X@{}}{
\begin{minipage}[t]{\linewidth}
    \begin{itemize}[nosep,after=\strut, leftmargin=1em, itemsep=3pt]
        \item[--] Coordinated and managed 9 training sessions for 450+ IT employees, utilizing Excel to monitor enrollment and attendance
        \item[--] Produced engaging explainer videos using Powtoon for each IT role to facilitate employee onboarding and training
        \item[--] Designed Mentimeter questionnaires for training sessions and analyzed live polling results to enhance training effectiveness
        \item[--] Collected, interpreted, and presented data from questionnaires in a visually compelling format to create reports
        \item[--] Maintained SharePoint site to disseminate communications throughout the organization
    \end{itemize}
    \end{minipage}
}\
\end{tabularx}
    
\begin{tabularx}{\linewidth}{ @{}l r@{} }
\textbf{Inventory/Merchandising/Product Flow Specialist} &  \hfill May 2021 - February 2022 \\
Best Buy | Timonium, MD \\[3.75pt]
\multicolumn{2}{@{}X@{}}{
\begin{minipage}[t]{\linewidth}
    \begin{itemize}[nosep,after=\strut, leftmargin=1em, itemsep=3pt]
        \item[--] Recorded and monitored inventory levels by completing supply audits daily to monitor discrepancies
        \item[--] Ensured optimal product availability on the sales floor by performing daily replenishment tasks
        \item[--] Collaborated with store staff to design visually appealing merchandise displays to enhance the customer experience
    \end{itemize}
    \end{minipage}
}\
\end{tabularx}

\begin{tabularx}{\linewidth}{ @{}l r@{} }
\textbf{Warehouse Associate} &  \hfill June 2020 - August 2020 \\
Amazon | Baltimore, MD \\[3.75pt]
\multicolumn{2}{@{}X@{}}{
\begin{minipage}[t]{\linewidth}
    \begin{itemize}[nosep,after=\strut, leftmargin=1em, itemsep=3pt]
        \item[--] Received and stocked pallets of inventory in room-temperature, chiller, and frozen sections of the warehouse
        \item[--] Picked between 80-100 units per hour to fulfill customer orders
        \item[--] Packed, crated, and tagged customer orders to prepare for shipment
    \end{itemize}
    \end{minipage}
}\
\end{tabularx}

%----------------------------------------------------------------------------------------
% COMMUNITY INVOLVEMENT SECTIONS
%----------------------------------------------------------------------------------------

%Community Involvement
\section{Community Involvement}

\begin{tabularx}{\linewidth}{ @{}l r@{} }
\textbf{Kappa Theta Pi, Professional Technology Fraternity} &  \hfill September 2021 - May 2023 \\
University of Maryland | College Park, MD \\[3.75pt]
\multicolumn{2}{@{}X@{}}{
\begin{minipage}[t]{\linewidth}
    \begin{itemize}[nosep,after=\strut, leftmargin=1em, itemsep=3pt]
        \item[--] Enhanced technical, professional, and interpersonal skills by actively participating in professional development events, technical workshops, community outreach, and social activities
        \item[--] Contributed to the Yearbook, Merchandise, and Philanthropy Committees by utilizing Adobe Premiere Pro for video production, Canva for yearbook creation, and Excel for collecting and validating  merchandise orders
    \end{itemize}
    \end{minipage}
}\
\end{tabularx}

\begin{tabularx}{\linewidth}{ @{}l r@{} }
\textbf{Climate Emergency Mobilization Work Group} &  \hfill September 2020 - September 2021 \\
Frederick, MD \\[3.75pt]
\multicolumn{2}{@{}X@{}}{
\begin{minipage}[t]{\linewidth}
    \begin{itemize}[nosep,after=\strut, leftmargin=1em, itemsep=3pt]
        \item[--] Created and sent out surveys to community members to receive feedback on climate education
        \item[--] Met with group members twice weekly to plan communication strategies and develop ideas for social media
        \item[--] Ran the official Instagram page by making posts, interacting with other social media pages, and hosting giveaways to gain traction with the community
    \end{itemize}
    \end{minipage}
}\
\end{tabularx}

\begin{tabularx}{\linewidth}{ @{}l r@{} }
\textbf{Carillon Communities} &  \hfill August 2019 - May 2020 \\
University of Maryland | College Park, MD \\[3.75pt]
\multicolumn{2}{@{}X@{}}{
\begin{minipage}[t]{\linewidth}
    \begin{itemize}[nosep,after=\strut, leftmargin=1em, itemsep=3pt]
        \item[--] Developed strong interpersonal skills in this living and learning program focused on weather and climate
        \item[--] Translated curiosity into action using creative problem solving and teamwork
        \item[--] Increased leadership skills working in teams
    \end{itemize}
    \end{minipage}
}\
\end{tabularx}

%----------------------------------------------------------------------------------------
% PROJECTS SECTIONS
%----------------------------------------------------------------------------------------

%Projects
\section{Projects}

\begin{tabularx}{\linewidth}{ @{}l r@{} }
\textbf{Crime in Prince George's County, MD} & \hfill \href{https://zcanast.github.io/crime-trends/}{Link to Demo} \\[3.75pt]
\multicolumn{2}{@{}X@{}}{
\begin{minipage}[t]{\linewidth}
    \begin{itemize}[nosep,after=\strut, leftmargin=1em, itemsep=3pt]
        \item[--] Developed a web page utilizing HTML, CSS, and JavaScript to analyze crime trends in Prince George's County, MD
        \item[--] Made calls to an API of reported crime incidents handled by the PG County Police Department, created an interactive bar chart with the Chart.js library, and included a map of where the incidents took place using the Leaflet library
    \end{itemize}
    \end{minipage}
}\
\end{tabularx}

\begin{tabularx}{\linewidth}{ @{}l r@{} }
    \textbf{Concussion Trends in the National Football League (NFL)} & \hfill \href{https://public.tableau.com/app/profile/zoey.anastasiadis/viz/ConcussionTrendsintheNFL2012-2015/ConcussionTrendsintheNFL2012-2015draft}{Link to Project} \\[3.75pt]
    \multicolumn{2}{@{}X@{}}{
    \begin{minipage}[t]{\linewidth}
        \begin{itemize}[nosep,after=\strut, leftmargin=1em, itemsep=3pt]
            \item[--] Used Tableau to analyze and present concussion trends in the NFL from 2012-2015
            \item[--] Discovered that certain positions are more vulnerable to concussions and those teams are more likely to lose games
        \end{itemize}
        \end{minipage}
    }\
    \end{tabularx}
    
\begin{tabularx}{\linewidth}{ @{}l r@{} }
\textbf{Restaurants Near Me} \\[3.75pt]
\multicolumn{2}{@{}X@{}}{
\begin{minipage}[t]{\linewidth}
    \begin{itemize}[nosep,after=\strut, leftmargin=1em, itemsep=3pt]
        \item[--] Utilized SQL to apply relational database design concepts and created a database of Michelin Guide restaurants
        \item[--] Enabled tourists worldwide to execute targeted queries and streamline their dining plans
    \end{itemize}
    \end{minipage}
}\
\end{tabularx}
    
    
\begin{tabularx}{\linewidth}{ @{}l r@{} }
\textbf{Statistical Analysis of World Happiness} \\[3.75pt]
\multicolumn{2}{@{}X@{}}{
\begin{minipage}[t]{\linewidth}
    \begin{itemize}[nosep,after=\strut, leftmargin=1em, itemsep=3pt]
        \item[--] Investigated factors that impact world happiness through statistical methods and visualization in Excel and R
        \item[--] Found that social support and freedom to make life choices have a strong positive correlation with happiness levels
    \end{itemize}
    \end{minipage}
}\
\end{tabularx}
    
    
\begin{tabularx}{\linewidth}{ @{}l r@{} }
\textbf{Course Prerequisite Flexibility} \\[3.75pt]
\multicolumn{2}{@{}X@{}}{
\begin{minipage}[t]{\linewidth}
    \begin{itemize}[nosep,after=\strut, leftmargin=1em, itemsep=3pt]
        \item[--] Leveraged pandas library in Python to manipulate and analyze data in a CSV file of course descriptions at UMD
        \item[--] Generated a list of how flexible UMD colleges are based on prerequisites for their courses
    \end{itemize}
    \end{minipage}
}\
\end{tabularx}

%----------------------------------------------------------------------------------------
%	EDUCATION
%----------------------------------------------------------------------------------------
\section{Education}
\begin{tabularx}{\linewidth}{@{}l X@{}}	

\textbf{University of Maryland} | College Park, MD &  \hfill \normalsize May 2023 \\

Bachelor of Science in Information Science & \hfill \\ %GPA: 3.62 \\ 

\end{tabularx}

%----------------------------------------------------------------------------------------
%	CERTIFICATIONS
%----------------------------------------------------------------------------------------
% \section{Certifications}
% \begin{tabularx}{\linewidth}{@{}l X@{}}	

% \textbf{Career Essentials in Data Analysis} | Microsoft and LinkedIn &  \hfill \normalsize July 2023 \\

% \end{tabularx}

%----------------------------------------------------------------------------------------
%	SKILLS
%----------------------------------------------------------------------------------------
\section{Skills}
\begin{tabularx}{\linewidth}{@{}l X@{}}

%\textbf{Languages} & \normalsize{Python, Java, SQL, R, HTML/CSS, JavaScript, MATLAB}\\
%\textbf{Technologies}  &  \normalsize{Excel, GitHub, Microsoft Office, Google Suite, WordPress, Tableau, Power BI, Gephi, Adobe Premiere Pro, Adobe Photoshop, Figma, Canva}\\ 
%\textbf{Environments}  &  \normalsize{Visual Studio Code, Eclipse, MySQL, RStudio, Jupyter Notebooks}\\

\textbf{Languages} & \normalsize
\tag{Python}
\tag{Java}
\tag{R} 
\tag{SQL}
\tag{HTML/CSS}
\tag{JavaScript}
\tag{Visual Basic}
\tag{MATLAB}
\\

\textbf{Technologies}  &  \normalsize{
\tag{Excel}
\tag{GitHub}
\tag{Microsoft Office}
\tag{Google Suite}
\tag{WordPress}
\tag{Tableau}
\tag{Power BI}
\tag{SharePoint}
\tag{Power Automate}
\tag{Adobe Premiere Pro}
\tag{Adobe Photoshop}
\tag{Canva}
\tag{Miro}
\tag{Open AI}
\tag{Figma}
\tag{Gephi}
\tag{Powtoon}
}\\

\textbf{Environments}  &  \normalsize{
\tag{Visual Studio Code}
\tag{Eclipse}
\tag{MySQL}
\tag{RStudio}
\tag{Jupyter Notebooks}
}\\

\end{tabularx}

\vfill
\center{\footnotesize Last updated: \today}

\end{document}
